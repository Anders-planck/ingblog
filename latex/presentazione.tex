\documentclass[8pt]{beamer}
\usepackage[utf8]{inputenc}

\usepackage{amsmath}
\usepackage{setspace}


\renewcommand{\baselinestretch}{1.5}
\include{theme/theme}


\title{ISA Blog}
\subtitle{Presentazione del progetto di Ingegneria del Software Avanzata}
\author{Jipwouo chiege planck Anders}
\date{A.A. 2023/2024}

\begin{document}



\begin{frame}
    \titlepage
\end{frame}

\begin{frame}
    \frametitle{Introduzione}
    \begin{itemize}
        \item \textbf{ISA blog} è una piattaforma di blogging open-source, costruita con tecnologie moderne
        dove gli utenti possono condividere i propri pensieri e interagire con gli altri utenti.
    \end{itemize}
\end{frame}

\begin{frame}{Requisiti}
    \textbf{Requisiti funzionali primari}, gli utenti possono:
    \begin{itemize}
        \item Registrarsi e accedere all'applicazione.
        \item Visualizzare una lista di articoli pubblicati.
        \item Visualizzare i dettagli di un articolo.
        \item Interagire cogli articoli pubblicati (commenti, like).
    \end{itemize}
\end{frame}

\begin{frame}{Requisiti}
    \textbf{Requisiti funzioniali derivati}, gli utenti possono:
    \begin{itemize}
        \item Aggiornare i dettagli del loro profilo.
        \item Creare nuovi articoli.
        \item Commentare gli articoli esistenti.
      \end{itemize}
    Security:
    \begin{itemize}
        \item la creazione di nuovi articoli e interazione gli altri utenti deve essere permessa solo agli utenti registrati,
        \item Gli utenti possono:
              \begin{itemize}
                  \item Registrarsi autonomamente.
                  \item Effettuare il login tramite dei provider authentication come email e github o google.
              \end{itemize}
    \end{itemize}
\end{frame}

\begin{frame}{Utenti}
    L'applicazione prevede due tipi di utenti:
    \begin{itemize}
        \item \textbf{Utente non registrato}: può visualizzare gli articoli e i commenti, ma non può creare nuovi articoli o commentare i articoli esistenti.
        \item \textbf{Utente registrato}: può creare nuovi articoli, commentare gli articoli esistenti e rispondere ai commenti.
    \end{itemize}
\end{frame}

\begin{frame}{Tecnologie e Liberie utilizzate}
    \begin{itemize}
        \item \textbf{Backend}: Supabase (un'alternativa open-source a Firebase) - un server database POSTGRES.
        \item \textbf{Hosting}: Vercel (una piattaforma di hosting per applicazioni web statiche e serverless)
        \item \textbf{Libreria UI}: Mantine (una libreria di componenti React)
        \item \textbf{Testing}: Jest (un runner di test veloce, leggero e con opinioni per progetti JavaScript moderni)
        \item \textbf{Containerisation}: Docker
        \item \textbf{Gestione dei stati e recupero di dati e cache}: Redux Toolkit \& RTK query
        \item \textbf{Forms}: React Hook Form (un form performante, flessibile ed estensibile con una validazione facile da usare)
    \end{itemize}
\end{frame}

\begin{frame}[allowframebreaks]
    \frametitle{Testing}
    Sono state testate le seguenti funzionalità:
    \begin{itemize}
        \item Tutte le pagine dell'applicazione sono state testate per verificare il corretto funzionamento.
        \begin{itemize}
            \item Home page: verifica del caricamento corretto dei articoli e della possibilità di navigare a un articolo specifico.
            \item Pagina del singolo articolo: verifica del caricamento corretto dei dettagli del post e dei commenti associati.
            \item Pagina di login/registrazione: verifica del corretto funzionamento dei form di autenticazione.
        \end{itemize}
        \item sono stati testati i servizi di Supabase per verificare il corretto funzionamento dell'API.
        \begin{itemize}
            \item Recupero dell'utente: verifica che i dettagli dell'utente vengano recuperati correttamente dall'API.
            \item Aggiornamento dell'utente: verifica che i dettagli dell'utente possano essere aggiornati correttamente tramite l'API.
            \item Recupero dei post: verifica che i post vengano recuperati correttamente dall'API.
            \item Creazione di un post: verifica che un nuovo post possa essere creato correttamente tramite l'API.
            \item Creazione di un commento: verifica che un nuovo commento possa essere creato correttamente tramite l'API.
            \item Recupero dei commenti: verifica che i commenti vengano recuperati correttamente dall'API.
        \end{itemize}
        \item sono stati testati i componenti React per verificare il corretto funzionamento.
        \begin{itemize}
            \item Componente Post: verifica che il componente Post renderizzi correttamente i dettagli di un post.
            \item Componente Comment: verifica che il componente Comment renderizzi correttamente un commento.
        \end{itemize}
        \item sono state effettuate dei unit tests su funzioni di trasformazione di dati e di utility.
    \end{itemize}
\end{frame}

\begin{frame}
    \frametitle{CI/CD}
    L'applicazione utilizza le GitHub Actions per il CI/CD.
    Sono stati definiti 5 job:
    \begin{itemize}
        \item \textbf{Su Master}:
        \begin{itemize}
            \item \textbf{Build}: viene eseguito il build dell'applicazione
            \item \textbf{Test}: vengono eseguiti i test dell'applicazione
            \item \textbf{Deploy}: viene eseguito il deploy della storybook su Vercel automaticamente
            \item \textbf{Deploy}: viene eseguito il deploy dell'applicazione su Vercel
            \item \textbf{Deploy}: viene buildata l'immagine Docker dell'applicazione e pubblicata su Docker Hub
        \end{itemize}
        \item \textbf{Su Develop}:
        \begin{itemize}
            \item \textbf{Build}: viene eseguito il build dell'applicazione
            \item \textbf{Test}: vengono eseguiti i test dell'applicazione
            \item \textbf{Deploy}: viene eseguito il deploy della preview dell'applicazione su Vercel
        \end{itemize}
        \item \textbf{Su Pull Request}:
        \begin{itemize}
            \item \textbf{Build}: viene eseguito il build dell'applicazione
            \item \textbf{Test}: vengono eseguiti i test dell'applicazione
        \end{itemize}
    \end{itemize}
\end{frame}

\end{document}
